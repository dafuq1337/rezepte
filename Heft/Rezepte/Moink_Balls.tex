\prorezeptheader{
	Moink Balls
}{
	Zurueckgeblieben
}{
	460901
}{
	\diffaltschwuchtel
}

\hspace{10mm}

\promisctext{Der Name dieses in den USA beliebten Finger Foods setzt sich aus ''Rind (muh) + Schwein (moink) = Moink'' zusammen.

\noindent Die Hackfleischbällchen werden aus Rinderhackfleisch gemacht und später mit Bacon umwickelt.}

% \noindent\proimportant{Vorsicht:} Suchtgefahr!}

\hspace{10mm}

\prorezeptzutaten{
	\item 500g Rinderhack
	\item 1 Zwiebel
	\item 1 Ei
	\item Paniermehl
	\item Hackfleischwürzer / eigene Gewürzmischung
	\item Chili
	\item Bacon
	\item Petersilie (nach Belieben)
	\item BBQ-Soße zum Bestriechen
	\item Optional: Mit Cheddar füllen
	\item Optional: In Tortilla-Bröseln wälzen
}
\prorezeptequipment{
	\item Zahnstocher
	\item Grill
	\item Vorteilhaft: Bratenthermometer

	\propostqrinline{460901}
}

\prorezeptzubereitung{
	\item Das Rinderhack zusammen mit gehackter Zwiebel, Ei, Paniermehl, nach Wunsch etwas Petersilie und den Gewürzen in eine Schüssel geben und zu einer homogenen Masse verkneten.
	\item Aus der Hackmischung nun kleine Bällchen formen. Nach Belieben mit Cheddar füllen (ist ne menge Arbeit!).
	\item Diese Bällchen mit Bacon umwickeln und mit einem Zahnstocher fixieren.
	\item Moink Balls werden indirekt gegrillt. Da heißt, dass das Grillgut im Gegensatz zum direkten Grillen nicht direkt über der Glut liegt. Die Hackfleishcbällchen nun bis ca. \produration{70\celsius} Kerntemperatur grillen, das prüft Ihr am besten mit einem Bratenthermomenter.
	\item Kurz vor dem Erreichen der Kerntemperatur die BBQ-Soße erwärmen und die Moink Balls damit bepinseln (und nach Belieben in Tortillacrunch toppen), um eine schöne, glänzende Optik zu erreichen.
	\item Diese nun je nach Wunsch bis zu einer Kerntemperatur von ca. \produration{75\celsius~bis 80\celsius} fertig grillen.
}

\promisctext{Wir haben nach Auge gegrillt, da keiner so ein Thermometer hat.}

\subsection{OC}
\prooc{460901-0}
