\prorezeptheader{
	Terence-Pfanne
}{
	Migame
}{
	428664
}{
	\diffftb
}

\hspace{10mm}

\prorezeptmenge{4}{1}

\hspace{10mm}

\prorezeptzutaten{
	\item 2 Zwiebeln
	\item 2 EL Öl
	\item 2 Dosen Flageolets (grüne Bohnenkerne, á 400g)
	\item 200g Creme Fraîche
	\item 1 Cabanossi
	\item Salz/Pfeffer
	\item Frische Petersilie
}
\prorezeptequipment{
	\item Pfanne
	\item Messer

	\propostqrinline{428664}
}

\hspace{10mm}

\promisctext{Dazu: Baguette, Bier}

\hspace{10mm}

\promisctext{Als Ergänzung zur \glqq Bud-Pfanne\grqq, hier die passende Terence-Pfanne! Flageolets sind grüne Bohnenkerne (sehr lecker und nicht mit grünen Stangenbohnen zu venwechseln!), die etwa aussehen wie grüne Kidneybohnen. Gibt es in jedem gut sortierten Supermarkt, geschmacklich die besten Bohnen (meine Meinung).}

\prorezeptzubereitung{
	\item Die Cabanossi in Scheiben schneiden, die Zwiebeln schälen, vierteln und anschließend ebenfalls in Scheiben schneiden.
	\item Öl in einer Pfanne erhitzen und die Cabanossi darin anbraten, anschließend die Zwiebeln dazugeben und glasig dünsten. Die Bohnenkerne wahrenddessen in einem Sieb abtropfen lassen und ebenfalls mit in die Pfanne geben. Unter gelegentlichem Umrühren erhitzen. Nicht anbrennen lassen.
	\item Die Crème Fraîche dazugeben und erhitzen, alles aufkochen lassen und mit Salz und Pfeffer nach Geschmack würzen.
	\item Die Petersilie hacken und kurz vor dem Servieren mit in die Pfanne geben.
}

\promisctext{Dazu passen Baguette und Bier!

\noindent Guten Appetit!}

% \subsection{OC}
% \prooc{}
