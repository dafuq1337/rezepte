\prorezeptheader{
	Mikrowellen-Kuchen 2.0
}{
	Bnkd
}{
	461595
}{
	\diffftb
}

\prorezeptzutaten{
	\item 60g Mehl
	\item 3/8TL Backpulver
	\item 25g Zartbitterschokolade
	\item 30g Butter
	\item Prise Salz
	\item 1 Ei
	\item 2EL Milch
	\item 1EL saure Sahne
}
\prorezeptequipment{
	\item 2 Tassen
	\item Rührschüssel
	\item Küchenwaage
	\item Schneebesen
	\item Mikrowelle (800 Watt Bass Machine)
	\item Etwas Geduld

	\propostqrinline{461595}
}

\prorezepttipps{
	\item Sofort spülen
	\item Schon weiche Butter benutzen
	\item \proimportant{Genau} wiegen
	\item Gras in Butter verarbeiten (hier keins dabei)
	\item Tassen sind heiß, dalso aufpassen
}

\prorezeptzubereitung{
	\item Als erstes mischen wir das Mehl mit dem Backpulver. \proimportant{Wichtig hier:} Sehr gut mischen, da sonst nichts daraus wird.
	\item Jetzt die Butter in die Rührschüssel geben und für \produration{30 Sekunden} in die Mikrowelle. In der Zwischenzeit wird die Schokolade kleingehackt oder durch eine Reibe gerieben. Wie Du an eine Reibe kommst, ist im Rezept ~\ref{sec:Feuerzangenbowle} beschrieben.
	\item Die Butter jetzt mit dem Zucker und dem Salz schaumig quirlen.
	\item Nun das Ei hinzugeben und wieder gut rühren.
	\item Wenn eine gleichmäßige Pampe entstanden ist, die Milch und die saure Sahne dazu geben und so lange rühren, bis eine homogene Masse entstanden ist.
	\item \proimportant{Jetzt erst} die Schokosplitter dazugeben (falls Du die nicht schon nebenbei weggezogen hast).
	\item Ein wenig rühren und ab mit der Pampe in zwei eingefettete Tassen. Die Tassen werden etwa halb voll gefüllt und die oberflöche ewas glattgestrichen.
	\item Für \proimportant{genau} \produration{1:50 Minuten} in die Mikrowelle bei 800 Watt. \proimportant{Jede Tasse einzeln} (wichtig, sonst explodiert alles und der Keller ist voller Senfgas).
	\item Die Tassen vorsichtig herausnehmen, da diese sehr heiß werden können. Etwas abkühlen lassen und auf einen Teller stürzen oder direkt aus der Tasse essen.
	\item Der ''Kuchen'' sollte sofort gegessen werden, da er kalt und alt nicht schmeckt.
}

\promisctext{Guten Appetit und verbrennt Euch nicht die Finger!}

\subsection{OC}
\proocsmall{461595-0}
\proocsmall{461595-1}
